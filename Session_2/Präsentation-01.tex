\documentclass[14pt,ngerman]{beamer}

\usepackage{babel}
\usepackage{siunitx}
\usepackage{tcolorbox}

\author{Uwe Ziegenhagen}
\title{Advances in Space}
\institute{Dante e.V. Heidelberg}
\date{Köln, den \today}
\logo{\includegraphics[width=1cm]{Bilder/Katze}}

\usetheme{Darmstadt}

\addtobeamertemplate{block begin}{%
  \setlength{\textwidth}{0.9\textwidth}%
}{}

\addtobeamertemplate{block alerted begin}{%
  \setlength{\textwidth}{0.7\textwidth}%
}{}

\addtobeamertemplate{block example begin}{%
  \setlength{\textwidth}{0.5\textwidth}%
}{}


\begin{document}

\begin{frame}
\maketitle

\end{frame}

\section{Einleitung}

\begin{frame}
\frametitle{Einleitung}
\framesubtitle{Literatur}

\begin{itemize}
	\item \SI{3.1345e20}{m^2}
	\item 
	\item 
	\item 
\end{itemize}

\end{frame}


\begin{frame}
\frametitle{Einleitung}
\framesubtitle{Literatur}

\begin{enumerate}
	\item \(a^2+b^2=c^2\)
	\item 
	\item 
	\item 
	\item 
	\item 
	\item 
	\item 
\end{enumerate}


\end{frame}

\begin{frame}
\frametitle{Einleitung}
\framesubtitle{Literatur}

\begin{center}
\includegraphics[width=0.8\textwidth]{Bilder/Katze}
\end{center}

\end{frame}


\begin{frame}
\frametitle{Einleitung}
\framesubtitle{Literatur}

\begin{columns}
\begin{column}{0.32\textwidth}
\begin{itemize}
	\item 
	\item 
	\item 
\end{itemize}
\end{column}
\begin{column}{0.32\textwidth}
\begin{itemize}
	\item 
	\item 
	\item 
\end{itemize}
\end{column}
\begin{column}{0.32\textwidth}

\includegraphics[width=\textwidth]{Bilder/Katze}

\end{column}

\end{columns}
\end{frame}

\begin{frame}
\frametitle{}

\begin{itemize}
\item<2> Hallo
\item<1-> ich 
\item<3-4>bin
\item<6> ein
\item<5> Text
\item der auf- und zugeht
\end{itemize}
\end{frame}

\begin{frame}
\frametitle{}

\begin{itemize}[<+->]
\item Hallo
\item ich 
\item bin
\item ein
\item Text
\item der auf- und zugeht
\end{itemize}
\end{frame}


\begin{frame}
\frametitle{}

\begin{block}{ffsd}
\begin{itemize}
\item 
\item 
\item 
\item 
\item 
\item 
\end{itemize}
\end{block}


\end{frame}


\begin{frame}
\frametitle{}


\begin{alertblock}{Aber Achtung!}

\(a^2 + b^2 ) c^2\)

\end{alertblock}
\end{frame}


\begin{frame}
\frametitle{}


\begin{exampleblock}{Beispiel}

\(a^2 + b^2 ) c^2\)

\end{exampleblock}

\begin{itemize}
	\item 
	\item 
	\item 
	\item 
	\item 
	\item 
	\end{itemize}

\end{frame}

\begin{frame}
\frametitle{\texttt{tcolorbox} Beispiel}

\(\Rightarrow\) erfordert tcolorbox Paket!

\begin{tcolorbox}
Hallo Welt!
\end{tcolorbox}

\end{frame}

\begin{frame}
\frametitle{}

\input{Verzeichnis.txt}


\end{frame}


\end{document}