\documentclass[12pt,ngerman,parskip=half]{scrartcl}

\usepackage[utf8]{inputenc}
\usepackage[T1]{fontenc}
\usepackage{booktabs}
\usepackage{babel}
\usepackage{graphicx}
\usepackage{csquotes}
\usepackage{paralist}
\usepackage{xcolor}
\usepackage{blindtext}

\begin{document}

\blindtext

Ich bin eine $a^2+b^2=c^2$ Formel im Fließtext. % TeX-Notation

Ich bin eine abgesetzte $$a^2+b^2=c^2$$ Formel. % TeX-Notation

Nehmt besser die \LaTeX-Notation!

Ich bin eine \(a^2+b^2=c^2\) Formel im Fließtext. % LaTeX-Notation

Ich bin eine abgesetzte \[a^2+b^2=c^2\] Formel. % LaTeX-Notation

\begin{equation}
- \frac{p}{2} \pm \sqrt{ \left(\frac{p}{2}\right)^2  -q  } \Rightarrow a^2 + b^2 = c^2
\end{equation}

\begin{equation}
- \frac{p}{2} \pm \sqrt{ \left(\frac{p}{2}\right)^2  -q  } \Leftrightarrow a^2 + b^2 = \Omega
\end{equation}

\begin{equation}
\sum_{i=1}^{\infty} x_{i_2} \times {x^2}^3
\end{equation}

Siehe Formel \ref{eq:eins}

\begin{equation}
\prod_{i=1}^{-\infty} x_{i_2} {x^2}^3 \cap \cup < 1 \geq 2 \not= 3 \leq 5 \label{eq:eins}
\end{equation}

\end{document}