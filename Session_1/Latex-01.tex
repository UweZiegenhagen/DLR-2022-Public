\documentclass[ngerman,12pt,parskip=half]{scrreprt} 

\usepackage[T1]{fontenc}
\usepackage{xcolor}
\usepackage{graphicx}

\usepackage{booktabs}
\usepackage{paralist}

\usepackage{babel}
\usepackage{blindtext}

\author{Uwe Ziegenhagen}
\title{Mein erstes \LaTeX-Dokument}

\usepackage{hyperref} % Als letztes Paket laden

\hypersetup{
    bookmarks=true,                     % show bookmarks bar
    unicode=false,                      % non - Latin characters in Acrobat’s bookmarks
    pdftoolbar=true,                        % show Acrobat’s toolbar
    pdfmenubar=true,                        % show Acrobat’s menu
    pdffitwindow=false,                 % window fit to page when opened
    pdfstartview={FitH},                    % fits the width of the page to the window
    pdftitle={My title},                        % title
    pdfauthor={Author},                 % author
    pdfsubject={Subject},                   % subject of the document
    pdfcreator={Creator},                   % creator of the document
    pdfproducer={Producer},             % producer of the document
    pdfkeywords={keyword1, key2, key3},   % list of keywords
    pdfnewwindow=true,                  % links in new window
    colorlinks=true,                        % false: boxed links; true: colored links
    linkcolor=blue,                          % color of internal links
    filecolor=blue,                     % color of file links
    citecolor=blue,                     % color of file links
    urlcolor=blue                        % color of external links
}

\newcommand{\person}[1]{\textcolor{red}{\textsc{\textbf{#1}}}}

\newcommand{\tier}[1]{\textcolor{magenta}{\textsc{\textbf{#1}}}}

\newcommand{\latexbefehl}[1]{\textcolor{blue}{\ttfamily\textbackslash #1}}

\newcommand{\zweiparams}[2]{\textcolor{#1}{\ttfamily\textbackslash #2}}

\begin{document} 
\maketitle

\tableofcontents

\chapter{Einführung}

\latexbefehl{documentclass}

\zweiparams{green}{documentclass}


\textcolor{red}{\textbf{Uwe Ziegenhagen}}

\person{Uwe Ziegenhagen}

\tier{Melly}

\begin{itemize}[\(\Rightarrow\)]
\item Hallo
\item ich bin ein Item
\item auf der Item Liste
\end{itemize}

\begin{compactitem}[\(\Rightarrow\)]
\item Hallo
\item ich bin ein Item
\item auf der Item Liste
\end{compactitem}


\begin{enumerate}[i]
\item Hallo
\item ich bin ein Item
\item auf der Enumerate Liste
\end{enumerate}

\begin{compactenum}[i]
\item Hallo
\item ich bin ein Item
\item auf der Enumerate Liste
\end{compactenum}


\begin{itemize}
\item Hallo
\item ich bin ein Item
\begin{enumerate}
\item Hallo
\item ich bin ein Item
\item auf der Enumerate Liste
\end{enumerate}

\item auf der Item Liste
\end{itemize}


\begin{description}
\item[Apfel] eine Frucht
\item[Birne] auch eine Frucht
\item[Pfirsich] noch eine weitere Frucht
\end{description}

\begin{compactdesc}
\item[Apfel] eine Frucht
\item[Birne] auch eine Frucht
\item[Pfirsich] noch eine weitere Frucht
\end{compactdesc}

\section{Schriftgrößen}

{\tiny Uwe}

{\scriptsize Uwe} 

{\footnotesize Uwe} 

{\small Uwe} 

{\normalsize Uwe} 

{\large Uwe} 

{\Large Uwe} 

{\LARGE Uwe} 

{\huge Uwe} 

{\Huge Uwe} 

\texttt{dir /?} \textsf{\bfseries Ein Text ohne Serifen}

\section{Literaturüberblick}

\blindtext[15]

\subsection[Literatur im deutschsprachigen Raum]{Literatur im deutschsprachigen Raum in der Zeit von Bismarck bis Grotewohl 1871--1960}

\blindtext[5]

\subsection{Literatur im englischsprachigen Raum}

\blindtext[5]

\subsubsection{Foo}

\blindtext[5]

\subsubsection{Bar}

\blindtext[5]

\paragraph{Uwe} \blindtext

\subparagraph{Uwe} \blindtext

\section{Aktueller Stand der Forschung}

\blindtext[5]



\chapter{Hauptteil}

\section{Forschungsschwerpunkt}

\blindtext[5]

\subsection{Physik}

\blindtext[5]

\subsection{Chemie}

\blindtext[5]

\subsubsection{Foo}

\blindtext[5]

\subsubsection{Bar}

\blindtext[5]

\paragraph{Uwe} \blindtext

\subparagraph{Uwe} \blindtext

\section{Aktueller Stand der Forschung}

\blindtext[5]

\chapter{Fazit}

\blindtext[25]

\end{document} 

