\documentclass[ngerman,12pt,parskip=half]{scrreprt} 
% draft als Option verhindert das Einbetten von Bildern, geht dann schneller

\usepackage[T1]{fontenc}
\usepackage{babel}

\usepackage{xcolor}
\usepackage{graphicx}

\usepackage{booktabs}
\usepackage{paralist}
\usepackage{csquotes}


\usepackage{blindtext}

\author{Uwe Ziegenhagen}
\title{Mein erstes \LaTeX-Dokument}

\usepackage{hyperref} % Als letztes Paket laden

\hypersetup{
    bookmarks=true,                     % show bookmarks bar
    unicode=false,                      % non - Latin characters in Acrobat’s bookmarks
    pdftoolbar=true,                        % show Acrobat’s toolbar
    pdfmenubar=true,                        % show Acrobat’s menu
    pdffitwindow=false,                 % window fit to page when opened
    pdfstartview={FitH},                    % fits the width of the page to the window
    pdftitle={My title},                        % title
    pdfauthor={Author},                 % author
    pdfsubject={Subject},                   % subject of the document
    pdfcreator={Creator},                   % creator of the document
    pdfproducer={Producer},             % producer of the document
    pdfkeywords={keyword1, key2, key3},   % list of keywords
    pdfnewwindow=true,                  % links in new window
    colorlinks=true,                        % false: boxed links; true: colored links
    linkcolor=blue,                          % color of internal links
    filecolor=blue,                     % color of file links
    citecolor=blue,                     % color of file links
    urlcolor=blue                        % color of external links
}

\newcommand{\person}[1]{\textcolor{red}{\textsc{\textbf{#1}}}}

\newcommand{\tier}[1]{\textcolor{magenta}{\textsc{\textbf{#1}}}}

\newcommand{\latexbefehl}[1]{\textcolor{blue}{\ttfamily\textbackslash #1}}

\newcommand{\zweiparams}[2]{\textcolor{#1}{\ttfamily\textbackslash #2}}

\newcommand{\redfootnote}[1]{\footnote{\textcolor{red}{#1}}
}


\begin{document} 
\maketitle

\tableofcontents

\listoffigures

\listoftables

\chapter{Einführung}

\latexbefehl{documentclass}

\zweiparams{green}{documentclass}

Siehe Abbildung \ref{fig:katze} auf Seite \pageref{fig:katze}, die sich in Kapitel \ref{ch:fazit} befindet.

\textcolor{red}{\textbf{Uwe Ziegenhagen}}

\person{Uwe Ziegenhagen}

\section{Fuß- und Randnoten}

\blindtext\marginpar{Ich bin Text im Rand und sollte nicht zu lang sein}

Sie sagte: \enquote{Hallo, ich \enquote{Hallo, ich lerne LaTeX} lerne LaTeX}

\blindtext\footnote{\textcolor{red}{\blindtext[2]}}


\tier{Melly}

\begin{itemize}[\(\Rightarrow\)]
\item Hallo
\item ich bin ein Item
\item auf der Item Liste
\end{itemize}

\begin{compactitem}[\(\Rightarrow\)]
\item Hallo
\item ich bin ein Item
\item auf der Item Liste
\end{compactitem}


\begin{enumerate}[i]
\item Hallo
\item ich bin ein Item
\item auf der Enumerate Liste
\end{enumerate}

\begin{compactenum}[i]
\item Hallo
\item ich bin ein Item
\item auf der Enumerate Liste
\end{compactenum}


\begin{itemize}
\item Hallo
\item ich bin ein Item
\begin{enumerate}
\item Hallo
\item ich bin ein Item
\item auf der Enumerate Liste
\end{enumerate}

\item auf der Item Liste
\end{itemize}


\begin{description}
\item[Apfel] eine Frucht
\item[Birne] auch eine Frucht
\item[Pfirsich] noch eine weitere Frucht
\end{description}

\begin{compactdesc}
\item[Apfel] eine Frucht
\item[Birne] auch eine Frucht
\item[Pfirsich] noch eine weitere Frucht
\end{compactdesc}

\section{Schriftgrößen}

{\tiny Uwe}

{\scriptsize Uwe} 

{\footnotesize Uwe} 

{\small Uwe} 

{\normalsize Uwe} 

{\large Uwe} 

{\Large Uwe} 

{\LARGE Uwe} 

{\huge Uwe} 

{\Huge Uwe} 

\texttt{dir /?} \textsf{\bfseries Ein Text ohne Serifen}

\section{Literaturüberblick}

\blindtext[15]

\subsection[Literatur im deutschsprachigen Raum]{Literatur im deutschsprachigen Raum in der Zeit von Bismarck bis Grotewohl 1871--1960}

\blindtext[5]

\subsection{Literatur im englischsprachigen Raum}

\blindtext[5]

\subsubsection{Foo}

\blindtext[5]

\subsubsection{Bar}

\blindtext[5]

\paragraph{Uwe} \blindtext

\subparagraph{Uwe} \blindtext

\section{Aktueller Stand der Forschung}

\blindtext[5]



\chapter{Hauptteil}

\section{Forschungsschwerpunkt}

\blindtext[5]

\subsection{Physik}

\blindtext[5]

\subsection{Chemie}

\blindtext[5]

\subsubsection{Foo}

\blindtext[5]

\subsubsection{Bar}

\blindtext[5]

\paragraph{Uwe} \blindtext

\subparagraph{Uwe} \blindtext

\section{Aktueller Stand der Forschung}

\blindtext[5]

\chapter{Fazit}\label{ch:fazit}

\blindtext[25]

\includegraphics[width=5cm]{Bilder/Katze.jpg}

\begin{center}
\includegraphics[width=0.8\textwidth]{Bilder/Katze.jpg}
\captionof{figure}{Meine Katze Melli}\label{fig:katze}
\end{center}

\clearpage % Seitenumbruch erzwingen

\section{Tabellen}

\begin{center}
\captionof{table}{Meine unschöne Tabelle}\label{tab:erstetabelle}
\begin{tabular}{|l|r|c|p{5cm}|} \hline
123 & Hallo Welt & abc & Hallo, ich bin ein kleiner Absatz \\ \hline
4545123 & Welt & abcdefg & Hallo, ich bin ein kleiner Absatz mit mehr Text \\ \hline
\end{tabular}
\end{center}

\begin{center}
\captionof{table}{Meine unschöne Tabelle}\label{tab:erstetabelle}
\begin{tabular}{lrcp{5cm}} \\ \toprule[2pt]
\textbf{Spalte 1} & \textbf{Spalte 2}  & \textbf{Spalte 3}  & \textbf{Spalte 4}  \\ \midrule[1pt]
123 & Hallo Welt & abc & Hallo, ich bin ein kleiner Absatz \\ 
4545123 & Welt & abcdefg & Hallo, ich bin ein kleiner Absatz mit mehr Text \\ \bottomrule[4pt]
\end{tabular}
\end{center}

\chapter{Gleitobjekte}

\blindtext[1]

\begin{figure}[h]
\begin{center}
\includegraphics[width=0.8\textwidth]{Bilder/Katze.jpg}
\caption{Meine Katze Melli}\label{fig:katzegl}
\end{center}
\end{figure}

\blindtext[5]

\begin{table}[htbp]
\begin{center}
\caption{Meine unschöne Tabelle}\label{tab:erstetabelle}
\begin{tabular}{lrcp{5cm}} \\ \toprule[2pt]
\textbf{Spalte 1} & \textbf{Spalte 2}  & \textbf{Spalte 3}  & \textbf{Spalte 4}  \\ \midrule[1pt]
123 & Hallo Welt & abc & Hallo, ich bin ein kleiner Absatz \\ 
4545123 & Welt & abcdefg & Hallo, ich bin ein kleiner Absatz mit mehr Text \\ \bottomrule[1pt]
\end{tabular}
\end{center}
\end{table}

\blindtext[10]

\end{document} 

h = here
t = top
b = bottom
p = ? (paragraph?)

